\documentclass[12pt]{article}

\usepackage[T1]{fontenc}
\usepackage[a4paper, left=3cm, right=2cm, top=3cm, bottom=2cm]{geometry}
\usepackage[dvipsnames]{xcolor}
\usepackage[toc]{glossaries}
\usepackage{authblk}
\usepackage{cite}
\usepackage{fontspec}
\usepackage{graphicx}
\usepackage{hyperref}
\usepackage{relsize}
\usepackage{setspace}
\usepackage{subcaption}
\usepackage{verbatimbox}
\usepackage[english, portuguese]{babel}
    \addto\captionsportuguese{\renewcommand*\bibname{Referências}}
    \addto\captionsportuguese{\renewcommand*\contentsname{Sumário}}

\newcommand\myshade{85}
\newcommand{\pprime}{\ensuremath{^{\prime}}}
\newcommand{\RNum}[1]{\uppercase\expandafter{\romannumeral #1\relax}}

\newcommand{\numberedref}[1]{\ref{#1} --- \nameref{#1}}
\renewcommand\Authsep{, }
\renewcommand\Authand{, }
\renewcommand\Authands{, e }
\providecommand{\keywords}[1]{%
  \small
  \textbf{\textit{\iflanguage{english}{Keywords}{Palavras-chave} ---}} #1%
}

\hypersetup{
    pdftitle   = {Controle de acesso baseado em política para aplicações de Fog e Edge Computing em redes IoT},
    pdfauthor  = {Pedro Santi Binotto},
    pdfsubject = {Resumo em Português},
    linkcolor  = black,
    citecolor  = black,
    urlcolor   = black,
    colorlinks = true,
    filecolor  = black,
    linktoc    = page
}%

\newglossaryentry{iot} {
  name={IoT},
  description={\textit{Internet of Things} (\textit{Internet das Coisas}) é a abstração conceitual referente à
  aproximação e integração entre objetos físicos e elementos da vida cotidiana e os recursos digitais disponibilizados
  pela rede global de computadores \cite{mattern2010}. O termo é frequentemente empregado em discussões tratando da automação
  ``computadorização'' de aspectos físicos do mundo moderno, como \textit{Smart Cities}, \textit{Smart
  Homes}/\textit{Home Automation}, etc..
  }
}

\newglossaryentry{wsan} {
  name={WSAN},
  description={
    \textit{Wireless Sensor and Actuator Network}: redes de dispositivos \textit{microeletrônicos}, tipicamente pequenos,
    de baixo custo e baixo consumo energético, que realizam e comunicam medições através de sensores embarcados \cite{akyildiz2002}.
  }
}

\newglossaryentry{fog} {
  name={Fog Computing},
  description={é um modelo de infraestrutura, processamento e comunicação
  computacional distribuído e descentralizado que visa oferecer facilidades e serviços que seriam tipicamente possibilitados por
  \textit{Data Centers} tradicionais (isto é, centralizados), aproximando o processamento de informação dos
  dispositivos geradores (\gls{wsan}) de dados e reduzindo o uso de banda para transporte de tais dados \cite{bonomi2012}.
  }
}

\title{Controle de acesso baseado em política para aplicações de \textit{Fog} e \textit{Edge Computing} em redes IoT}
\author[1]{Pedro Santi Binotto [20200634]\thanks{\texttt{pedro.binotto@grad.ufsc.br}}}
\author[1]{Prof. Carlos Becker Westphal \thanks{\texttt{carlosbwestphall@gmail.com}}}
\date{\today}
\affil[1]{Departamento de Informática e Estatística, Centro Tecnológico \protect\\ Universidade Federal de Santa Catarina}

\makeglossaries

\begin{document}
\begin{titlepage}
\selectlanguage{portuguese} 
\maketitle
\thispagestyle{empty}

\begin{abstract}
  Em anos recentes, tem-se evidenciado um crescente interesse sobre a adoção de \textit{\gls{fog}} em
  cenários de computação distribuída e redes \gls{wsan}. Esta mudança de paradigma --- do tradicional modelo
  centralizado de \textit{Cloud Computing} para sistemas \textit{Edge} e \textit{Fog} --- salienta a importância de manter
  fundamentos sólidos para o desenvolvimento de mecanismos de segurança voltados à integridade e confiabilidade das redes
  de dispositivos \gls{iot} à medida que essas soluções extrapolam o contexto industrial e passam a ser implantadas em
  demais áreas críticas para o funcionamento da sociedade moderna. Este artigo propõe a possibilidade
  da aplicação de estratégias de proteção testadas e aprovadas em diversos domínios clássicos da computação distribuída
  para soluções \textit{Fog} e \textit{Edge}, demonstrando como estas podem ser implantadas de forma eficiente para
  manter uma rede coesa, íntegra e segura.
\end{abstract}

\selectlanguage{english} 
\begin{abstract}
  In recent years, a growing interest to the adoption to Edge and \gls{fog} in the domain of distributed computing has
  has come to the foreground. This paradigm shift --- from the classic, centralized solution of Cloud-based systems to
  Fog and Edge models --- highlights the need for a solid conceptual foundation for the development and implementation
  of security systems in \gls{iot} networks, as such systems have become increasingly common not only in industrial
  contexts but also in other critical domains essential to modern society. This paper proposes the application of
  established protection strategies from classical distributed computing domains to Edge and \gls{fog} environments to
  ensure the integrity and soundness of this type of network.
\end{abstract}

\selectlanguage{portuguese}

\keywords{\textit{Fog Computing}, \textit{Fog-Node Security}, IoT, PBAC, Segurança de Redes WSAN}

\end{titlepage}

\tableofcontents

\printglossary[title=Glossário, toctitle=Glossário]

\section{Introdução} \label{sec:intro}
\subsection{Motivação e Justificativas}
\subsubsection{Desafios e Obstáculos para a implantação de redes \textit{Fog}}
\paragraph{}
Desde a introdução inicial do conceito pela CISCO em 2012 \cite{bonomi2012}, o interesse em redes de \gls{fog} tem
sido um tópico de interesse crescente para pesquisadores, firmas de desenvolvimento de software, companhias do setor
industrial e iniciativas públicas e privadas com foco no progresso digital e automação de linhas de produção, meios de
transporte urbanos, veículos autônomos, etc. \cite{srirama2024}. 

A taxa de adoção de redes desta espécie, no entanto, não corresponde ao crescente nível de interesse na
aplicação da tecnologia descentralizada --- para a implantação efetiva de dispositivos na configuração \textit{Fog} e
\textit{Edge}, existem desafios econômicos, logísticos, e, como será explorado neste artigo, também obstáculos técnicos
à serem solucionados \cite{srirama2024}.

\subsubsection{Riscos de segurança na introdução de \textit{Fogs} em soluções distribuídas}
\paragraph{}
A característica definitiva que diferencia o modelo \textit{Fog} da computação em \textit{Cloud} tradicional é a aproximação
(física assim como topológica) do processamento de dados dos dispositivos de campo (\gls{wsan}). Desta forma, uma maior
parcela do trabalho computacional é realizada ``nas bordas'' (\textit{Edge}) da rede, e, consequentemente, a integridade
da rede como um todo também é atrelada à integridade da infraestrutura da rede local.

Por este motivo, redes de \gls{fog} são mais suscetíveis à alguns tipos de ameaças do que implantações \textit{Cloud}
tradicionais \cite{alhazmi2019,srirama2024,khan2017}:
\begin{itemize}
  \item Critérios fracos para autenticação de nós (\textit{nodes}) na rede;
  \item Propensão à ataques físicos (\textit{tampering, jamming}) sobre os dispositivos \gls{iot};
  \item Medidas de segurança e criptografia reduzidas como consequência de  fatores físicos dados pelo hardware limitado
    dos dispositivos que compõem a rede;
  \item Agravando a maior dependência sobre infraestrutura local, também sofre de uma maior superfície (maior número de equipamentos)
    elegível para ataques e acessos maliciosos;
\end{itemize}

Assim, torna-se necessária a análise de alternativas para desenvolver e reforçar métodos e estratégias para garantir a
segurança e integridade destas redes.

\subsection{Objetivos Gerais e Específicos}
\subsubsection{Um \textit{Framework} pragmático para a definição de confiabilidade}
\paragraph{}
Conteúdo de parágrafo.

\subsubsection{Uma proposta de aplicação conceitual de PBAC para redes \textit{Fog}}
\paragraph{}
Conteúdo de parágrafo.

\subsection{Organização do Artigo}
\paragraph{}
Este artigo está organizado da seguinte maneira:
\begin{itemize}
  \item \numberedref{sec:intro}: Apresenta o contexto inicial \dots;
  \item \numberedref{sec:concepts}: Introduz os componentes principais \dots;
  \item \numberedref{sec:correlated}: Apresenta trabalhos correlatos \dots;
  \item \numberedref{sec:aspects}: Só Deus sabe \dots;
  \item \numberedref{sec:problems}: Novamente não tenho certeza do que esse cara quer \dots;
  \item \numberedref{sec:solutions}: Idem \dots;
\end{itemize}

\section{Conceitos básicos} \label{sec:concepts}
\subsection{Autenticação, Controle de Acesso, Privacidade e PMAC}
\subsubsection{Fundamentos e \textit{Framework} Conceitual}
\paragraph{}
Conteúdo de parágrafo.

\subsection{\textit{TCB}, \textit{TEE} e \textit{Fog-Node Security}}
\subsubsection{\textit{CIA}}
\subsubsection{\textit{Chain-of-trust}}
\paragraph{}
Conteúdo de parágrafo.

\section{Trabalhos Correlatos} \label{sec:correlated}
\subsection{Primeiro trabalho correlato}
\paragraph{}
Conteúdo de parágrafo.

\subsection{Segundo trabalho correlato}
\paragraph{}
Conteúdo de parágrafo.

\section{Aspectos relevantes} \label{sec:aspects}
Conteúdo de parágrafo.

\section{Problemas existentes} \label{sec:problems}
Conteúdo de parágrafo.

\section{Soluções Possíveis} \label{sec:solutions}
Conteúdo de parágrafo.

\newpage
\section{Bibliografia}
\bibliographystyle{apalike}
\bibliography{references}

\end{document}

